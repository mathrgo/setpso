\documentclass[a4paper,oneside,english]{article}

\usepackage[sc,osf]{mathpazo}
\usepackage{units}
\usepackage{amsmath}
\usepackage{amsfonts}
\usepackage{amssymb}
\usepackage{amsthm}
\usepackage{stmaryrd}

%\makeatletter
%%%%%%%%%%%%%%%%%%%%%%%%%%%%%% Textclass specific LaTeX commands.
\numberwithin{equation}{section}
\numberwithin{figure}{section}
\numberwithin{table}{section}

%\makeatother

\usepackage{babel}
\begin{document}
	\title{Experiments in Finding 42 as Sum of 3 Cubes }
	\author{Martin Gate}
	\maketitle
\section{Introduction}
The sum of 3 cubes of signed integers has some interesting properties as is summarised in \cite{3-cubes-sum}. In particular for numbers $k$ up to 100 with $k \neq \pm 4 \mod 9$, after finding the solution  for 33 given in  \cite{sum-3-cubes-33}, there is only 42 as the outstanding number that is not known to be the sum of three cubes. This article gives some thoughts on the constraints for the $k=42$ case and how to encode into a general positive integer. Let \begin{equation}\label{cubes-42}
x_0^3+x_1^3+x_2^3 =42
\end{equation}
in the equation we are trying to solve for integers $x_0,x_1,x_2$.

\section{Interesting modular arithmetic results}
Taking the mod of a cube provides some interesting results:

\begin{table}[h]
	\caption{Modular Arithmatic Results}
	\begin{center}
		\begin{tabular}{|c||ccccccccc|c|}\hline
&\multicolumn{9}{c|}{$x$}&\\
 &1&2&3&4&5&6&7&8&9&42\\
\hline \hline
$x^3 \mod 7$& 1&1&-1&1&-1&-1&0&&&0 \\
$x^3 \mod 6$& 1&2&3&4&5&0&&&&0\\
$x^3 \mod 9$& 1&-1&0&1&-1&0&1&-1&0&-3\\
$x^3 \mod 3$& 1&-1&0&1&-1&0&1&-1&0&0\\
\hline	
\end{tabular}
	\end{center}
\end{table}
from the $\mod 7$ table one can see that one of the integer powers say $x_3$ is divisible by 7 so put $x_0 =7a$. Also note from the $\mod 9$ table that we have, since $42 \mod 9 \equiv -3$
\begin{equation}\label{m9}
X_i^3 \mod 9 \equiv -1\quad  \mathrm{for}\:  i= 0,1,2
\end{equation}  
 which in turn gives 
\begin{equation}\label{m3}
X_i^3 \mod 3 \equiv -1 \quad \mathrm{for}\:  i= 0,1,2
\end{equation}
so we get for $i= 0,1,2$
\begin{align}
x_i \mod 6 \equiv & x_i^3  \\
\equiv & 2 \mathrm{or} 5 \\
\equiv & 2 \mathrm{or} -1 \label{m6-options}
\end{align}
since $42 \mod 6 \equiv 0$ we get 
\begin{align}
x_0+x_1+x_2 \mod 6 \equiv & x_0^3+x_1^3+x_2^3 \\
\equiv & 0 \label{sum-m6}
\end{align}
also note that 
\begin{align}
x_0 \mod 6 \equiv & 7a\\
\equiv & (7 \mod 6)\:a\\
\equiv & a \label{a_m6}
\end{align}
We also get from the $\mod 7$ table since $x_0 \equiv 0 \mod 7$ that \begin{align}
x_1^3 \mod 7 \equiv & -x_2^3 \\
\equiv & \pm1
\end{align}
so without loss of generality we put\begin{align}
x_1^3 \mod 7 \equiv & 1 \label{m7-x1}\\ 
x_2^3 \mod 7 \equiv & -1 \label{m7-x2} 
\end{align} 
\section{Forming Candidates from a big integer  }
When working with a Set Partical Swarm Optimiser (SPSO) the cost function is presented with a random big integer that represents in coded form the candidate to cost after it has been modified to meet given constraints.I believe The adoption of the modular constraints avoids attracting the SPSO towards low scoring solutions away from the solution we are here interested in. This section describes how this is done for  Equation \ref{cubes-42}.

\subsection{partitioning the big integer} 
For  SPSO the parameters big integer is a positive one and can be regarded as an array of bits with least significant bits to the left of the array in our representation; using this the big integer is partitioned as\footnote{In coding this up the positive big integer is represented as an array of 64 bit words; for the sake of computational speed the partition is applied at word boundaries. }
\begin{equation}\label{big-int}
b=j_0|j_1|j_2|f
\end{equation}
where $j_i i=0,1,2$ are positive integers occupying the same number of bits say $N$ with possible padding with zeros on the right as represented. $f$ is regarded as an array of flags taking on the value 1 or 0.

From this we get three signed integers
\begin{equation}\label{sighed-int}
k_i=-1^{f[i]} j_i \quad \mathrm{for} i=0,1,2
\end{equation}  
that provide a starting point in representing the candidate integers for \ref{cubes-42} by modifying the $k_i$
\begin{align}
a = &6 k_0 +c_0\\
x_1= &6 k_1+c_1\\
x_2= &6 k_2+c_2
\end{align}
where the $c_i$ are yet to be chosen based on $f$ to meet the $\mod 6$ constraints.

Once the $k_i$ are found the the $j_i$ are replaced by $|k_i|$ and the $f[i]$ replaced by $\mathrm{sign}(k_i)$ if required to give the modified big integer that gives the constraint satisfying parameters to be used in the next iteration in the SPSO.
\subsection{Choosing the $c_i$ to satisfy the $\mod 6$ Constraints}
From \ref{a_m6}and \ref{m6-options} we have $c_0$ is either 2 or -1 so use $f[3]$ to choose the option and put
\begin{equation}\label{c-0}
c_0=\begin{cases}
2 & \mathrm{if}\quad  f[3] = 1\\
-1 & \mathrm{otherwise}
\end{cases}
\end{equation} 
the $x_1,x_2$ must satisfy \ref{m6-options} as well as \ref{sum-m6}. It transpires by inspection that $x_1$ can take on either $\mod 6$ option which in turn determines the $x_2$ options; to this extent put
\begin{equation}\label{c-0}
c_1=\begin{cases}
2 & \mathrm{if}\quad  f[4] = 1\\
-1 & \mathrm{otherwise}
\end{cases}
\end{equation} 
and then we have that $c_2$ satisfies the constraint \ref{sum-m6} if we use the table \ref{c2-values}.
\begin{table}[h]	\caption{\label{c2-values}$c_2$ Values}
	\begin{center}
	\begin{tabular}{|cc||c|}
	\hline
	$f[3]$&$f[4]$&$c_2$ \\
	\hline\hline
	1&1&2\\
	1&0&-1\\
	0&1&-1\\
	0&0&2\\
	\hline
\end{tabular}
\end{center}
\end{table}
 
\subsection{Changing the $k_i$ to Satisfy the $\mod 7$ Constraints}
Finally to satisfy the $\mod 7$ constraints in \ref{m7-x1} and  \ref{m7-x2} we shift the $\mod 7$ values by adding or subtracting to $k_1$ and $k_2$; this does not change the constants $c_i$ and it moves the $\mod 7$ results in reverse order. Since 
\begin{align}
6(x+d)+m \mod 7 \equiv &(6x+m) +6d \\
\equiv & (6x+m) -1d\\
\equiv & (6x+m) -d
\end{align} 
Let the substitutions be 
\begin{align}
k_1 \leftarrow & k_1+d_1\\
k_2 \leftarrow & k_2+d_2
\end{align}
then by inspection of the $\mod 7$ table we can satisfy the constraints  \ref{m7-x1} and \ref{m7-x2} using table \ref{d-values}
\begin{table}[h] 	\caption{\label{d-values} $d_1, d_2$ Values to satisfy \ref{m7-x1} and \ref{m7-x2}  }
	\begin{center}
		\begin{tabular}{|c||ccccccc|}
			\hline
			$x_1 \mod 7$&0 &1 &2 &3 &4 &5 &6 \\
			$d_1$       &-1&0 &0 &1 &0 &1 &2\\
			\hline	\hline
			$x_2 \mod 7$&0 &1 &2 &3 &4 &5 &6 \\
			$d_2$       & 1&-2&-1& 0&-1&0 &0\\
			\hline
		\end{tabular}
	\end{center}
\end{table} 



\bibliographystyle{alpha}
\bibliography{AiPlayC.bib} 
\end{document}