\documentclass[a4paper,oneside,english]{book}

\usepackage[sc,osf]{mathpazo}
\usepackage{units}
\usepackage{amsmath}
\usepackage{amsfonts}
\usepackage{amssymb}
\usepackage{amsthm}
\usepackage{stmaryrd}

%\makeatletter
%%%%%%%%%%%%%%%%%%%%%%%%%%%%%% Textclass specific LaTeX commands.
\numberwithin{equation}{section}
\numberwithin{figure}{section}

%\makeatother

\usepackage{babel}
\begin{document}
	\title{Experiments in Developing Discrete Particle Swarm Optimisers}
	\author{Martin Gate}
	\maketitle
	\tableofcontents\newpage
	
	\part[Algorithms]{Algorithm Description and Justification}
	\chapter{Optimiser to Function Split}
	\section{Setting the  Scene}
	In many cases an organism in nature has a goal to survive well in a changing environment it does this by actuating various limbs under the control of a brain that is trying to get a good outcome. The degree of effort and lack of a good outcome  can be regarded as a \emph{cost} that depends on the  environment  and sequence of actions that it has under its control. The brain is looking for a \emph{program}  to give good results most of the time so that it can respond effortlessly to changes of environment. Sometimes the program gives a costly outcome and the organism must modify the program. I believe such tweaking is done by random choice anything else such as deductive reasoning is really part of the program. Of course such choice will be biased towards cases that worked in the past. As such the  program to outcome cost is like a \emph{function } but with random fluctuation since the  same program may produce different outcomes depending on the environment. The \emph{program} in this context can be regarded as \emph{input} to the function to give a cost while the directed random choice is to be regarded as an \emph{optimiser}; by its nature the optimiser cannot see into the workings of this function otherwise it would be part of the function; all it has is the cost associated with the input as program and the constraint that the chosen input is indeed a working program. the rest of this book will be exploring the relation between function and optimiser, generating algorithms to implement them and apply  to hopefully interesting  cases that are models of environments that reflect real world behaviour.
	
	\chapter{Set Particle Swarm Optimiser}
\section{Introduction}
\subsection{Continuous Case model}
Particle Swarm Optimisers (PSO) introduced in \cite{KE-pso} have been developed to optimise problems with continuous tuning parameters. The PSO consists of a finite collection (called a swarm) of particles with current state (often called \textit{Parameters}) $x_t^i$ personal best state $p_t^i$ and velocity $V_t^i$; where $i$ is the particle index and $t$ is the iteration index. These three properties of each particle are updated at every iteration using 
\begin{align}
V_{t+1}^i&=&\mu \left(x_t^i,V_t^i,N_t^i\right)\\
x_{t+1}^i&=&\xi \left(x_t^i,V_t^i\right)\\
p_{t+1}^i&=&\left\lbrace \begin{array}{cc}
p_t^i& \mathrm{if}\; x_{t+1}^i\notin \varOmega \;\mathrm{or}\; F(p_t^i)<F(x_{t+1}^i)  \\ 
x_t^i&\mathrm{otherwise} 
\end{array}\right. 
\end{align} 
where $F(x)$ is the cost of the state $x$; $N_t^i$ is the  collection of personal best states used by the $i$th particle to update its velocity from iteration $t$ to $t+1$: $\varOmega$ is a feasible  subset of states usually characterised as satisfying a collection of constraints . 

The PSO has several variants and the above description is a general summery where for continuous parameters the state is a $d$ dimensional vector space of real numbers $\mathbb{R}^d$. Typically the velocity is made up from random multiples of the vector diference between $x_t^i$ personal bests and members of$N_t^i$ giving the swarms directions to go to look for improved $F(x)$ values. Also $\mu$ and $\xi$ are not true functions since they depend on hidden random variables that are crucial for ensuring the ability to find a low cost near optimal personal best after sufficient iteration, but are otherwise processes that depend on the variables given as their arguments.
 
\subsection{The direction of Interest}
The main interest is developing an optimiser for Machine Learning that bridges the gap between neural nets and more structure determining optimisers such as genetic algorithms. To do this the current work looks at combinatorial optimisation problems (COP) using discrete tuning parameters in the form of binary strings (BS) for particle state with a range of cost-function value types. Such  a PSO will be called a Set PSO ( abbreviated to SPSO) to emphasise that it interprets BS as  sub sets to represent the parameters of SPSO. 

\subsection{Machine Learning Extensions }
For  Machine Learning there is a need to extend the basic SPSO to cope with:
\begin{enumerate}
	\item noisy data mainly due to a restricted amount of available data when evaluating a cost-function. 
	\item represent continuous state such as design or neural net parameters by a set of discrete values some what similar to the way genetic algorithms deal with this by digitizing the continuous case.
	\item treat random strings of bits as a way of generating useful programs  to find optimising solutions which in turn can be used as a way of tuning meta parameters.
	\item Acknowledge need to avoid over fitting and that only approximate solutions are needed. 
\end{enumerate}  
\subsection{Implementation and Development}
The SPSO is developed mainly through biased intuitive analogy with  other systems. As such it needs a programming language to explore empirically its properties and implement it. The language chosen is Golang. The treatment of SPSO here attempts to make the description language agnostic, however sometimes golang constructs such as interface are directly used to simplify the description of algorithms and help the user understand what the code is trying to do.     

\section{Set PSO}
\subsection{Cost-function Value Representation}   
COP problems are discrete in nature and often have an integer cost-function value, because of this big integer cost values was initially used for SPSO.  The restriction to big integer is strictly not required since the cost function is used only to compare the cost between solutions so the cost function value in the software is represented by the interface type \texttt{CostValue}. This has the additional advantage that it can contain state to help deal with noisy data that needs to be evaluated several times to reduce the noise and store its variance.

Formaly \texttt{CostValue} object \texttt{cost} has the following interface methods were \texttt{x} is to be kept general to avoid a circular definition, but is expected to  have a type that can be converted to the cost value object being used; In this case \texttt{x} is a raw cost issued by the cost function  :
\begin{list}{}
	\item \texttt{cost.Set(x interface\{\})} this sets the object to a copy of the value \texttt{x}. 
	\item \texttt{cost.Update(x interface\{\})} this combines \texttt{x} with the internal state stored by \texttt{cost} usualy to generate a cost with reduced variance to aid comparison with other costs.
	\item \texttt{cost.Cmp(x interface\{\}) int} this compares the cost value object with x and returns the following values:
		\begin{list}{}
		\item $-2$ \texttt{cost} needs to be further evaluated to establish a comparison.
		\item $-1$ \texttt{cost} has a lower cost than \texttt{x}.
		\item $\quad 0$ \texttt{cost} has the same cost value as \texttt{x}.
		\item $\quad 1$ \texttt{cost} has a larger cost vale than \texttt{x}.
		\item $\quad 2$ \texttt{x} needs to be further evaluated to establish a comparison.
		\end{list}
	
	\item \texttt{cost.Fbits() float64} this returns a logarithmic like floating point value for plotting; it attempts to give the number of bits  used in the big integer case for positive integer. A zero cost returns $0$. To illustrate, this function for continuous deterministic  case  would be:
	\[ Fbits=\mathrm{sign}(cost)\log_2(1+|cost|)   \]

\end{list}
As can be seen the underlying implementation of the cost value is well hidden from SPSO by using this interface it allows the cost-function to have a range of implementations  of its cost value which could include big integers, floating point, or even strings representing programs! 

Here we talk about a cost value as if it has a numerical value; this may not be  the case it could for instance be a program to implement changes to reach a goal; in this case  \texttt{Fbits()} just gives an indication of the complexity of the changes while \texttt{Comp()} does the comparison between solutions. 

\texttt{Comp()} has two additional values $\pm2$ which are used in cases where the cost value is difficult to compare without further evaluation due to randomness in the costs being evaluated. behind the scenes the cost value could for instance have a measured mean and variance associated to the cost value. If there is too much variance associated with a solution pair there cannot be a meaningful comparison between them and further cases are needed to reduce the variance through more cost-function evaluation.  

Big integer cost values are used to include some of the difficult cases in combinatorial optimisation, although the chance of finding global best solutions to these problems is minimal; however, The algorithms are expected to produce good useful approximations to these hard problems; I would like to suggest that main area of application is \emph{machine learning where one is interested in good approximate minimisation solutions of difficult problems}.


\subsection{The Importance of Satisfying Constraint on Solutions }
COP problems such as the Travelling Sales Person(TSP) have constraints defining $\varOmega$ that must be satisfied if to be of any use so one must continuously ensure the candidate solution is in $\varOmega$ before evaluating its cost. Often feasible points $\varOmega$ do not form an open subset of the space of states so a raw update of $x_t^i$ will not give a feasible state most of the time. To cope with this $\xi$ computes a raw update $\breve{x}_{t+1}^i$ which it then attempts to  convert into a feasible one before being returned as a update. Sometimes this may not be possible and the returned $x_t^i$ is not in $\varOmega$. As indicated in the update equation this results in no change to the personal best thus ensuring that personal bests are updated to only feasible states.   To acheive a working alternative  in $\varOmega$ the  cost function does this. 

\subsection{Generalised Link Between COP and SPSO}
The link between COP and SPSO is forged by using binary strings to represent candidate algorithms which are obtained by decoding the string. I use the term algorithm to represent any method for finding approximate solutions to a COP problem including just providing an approximate solution with no use of an algorithm. I think decoding to an algorithm rather than just a solution gives the PSO significantly more expressive power to find approximate solutions. The structure of candidate solutions may not be the same as binary strings  so several binary strings may map to the same solution. The resulting solution is then evaluated by the cost-function to give its cost value indicating how well the algorithm fits; this could include some cost component representing the complexity of used algorithm; lower values being preferred to higher values as determined by \texttt{Comp()}. This procedure gives a mapping from a binary string to  cost  value of the binary string. The SPSO is used to learn a low cost global solution by using a swarm of proposed solutions to iteratively search for a low cost one.


Even in the case of a deterministic  function it may be expedient to cut down on the evaluation time by only partially evaluating it thus introducing uncertainty. For instance the solution string may represent a \emph{program} that has to be tested on a ridiculously large number of cases to give a perfect cost value in this case the cases are randomly chosen to test the solution and thus introducing uncertainty which is traded against speed of evaluation. Once the cost value uncertainty is measured the cost-function can determine whether two solutions can be compared and feed this back to the SPSO via \texttt{Comp()}.  

For the moment we restrict the SPSO state to be at most $N$ bits. Each binary string $x$ can thus be regarded as a subset of integers in the range $1 \cdots N$  where $j \in x$ when $x[j] = 1$. To this extent each non zero bit represents the inclusion of some element with a feature and as such encodings that favour this interpretation is expected to do better using the following algorithms. 

\subsection{Reading Across from the Continuous Case}
In the following let $rand$ be a random number generator that produces a number in the range $0\leq rand < 1$ for each use of it.Also let $\neg$ be the logical negation and $\leftarrow$ the assignment of left side to the right side of the arrow.

\subsubsection{Semantic Reinterpretation of Notation }
The notation for the continuous case can be now used where $x_t^i$ is just a subset rather than a vector. The velocity is interpreted as a vector of probabilities  where the $j$th component $V_t^i[j]$ is just the probability that $x_t^i[j]$ will flip during the update going from 0 to 1 or \textit{vice versa}.
\subsubsection{Adding up velocity contributions}
The velocity is built up from several contributions, which will be described in detail below, so there is a need for a way of adding up the contributions. the velocity is an array of probabilities each component being regarded as independent so one is interested in how to add probabilities. direct adding of probabilities does not work because this can give values greater than 1; multiplying probabilities just leads to a value smaller  than the probabilities being combined. A possible compromise is to combine the probabilities  $p_1$ and $p_2$ as follows:
\begin{equation}\label{comb_probs}
p_1\dot{+}p_2= p_1+p_2-p_1p_2
\end{equation}
and for two velocities we define
\begin{equation}\label{vel-add}
(V_1\dot{+}V_2)[j]=V_1[j]\dot{+}V_2[j] \qquad \forall j
\end{equation}       
this is the method adopted although not the only one that could have been used. For instance taking the maximum value of the probabilities is popular and could be an alternative. the chosen operation \eqref{comb_probs} is called \emph{pseudo-addition} since it has most properties of addition. This can be seen  by putting 
\begin{equation}\label{add-map}
p_i=1-e^{-\xi_i}
\end{equation}
and then noting that
\begin{align}
p_1\dot{+}p_2&=(1-e^{-\xi_1})+(1-e^{-\xi_2})-(1-e^{-\xi_1})(1-e^{-\xi_2})\\
&=1-e^{-\xi_1}e^{-\xi_2}\\
&=1-e^{-(\xi_1+\xi_2)}
\end{align} 
so the mapping $p_1 \mapsto \xi_i$ is a homomorphism between $\dot{+}$ and  $+$. In particular the pseudo-sum is commutative and associative.
\subsubsection{Velocity Components for Directing Towards $N_t^i$}
As for continuous case the velocity, as an indicator of the direction to go to find better solutions, is roughly increased along the diference between $x_t^i$ and $p_t^k$ for each $k \in  N_t^i$. For the SPSO the diference is represented by the exclusive or of the corresponding subsets giving the difference:
$$del_t^{(i,k)}= x_t^i\oplus p_t^k$$ 
From this a velocity increment can be produced  as $Vdel_t^{(i,k)}$ where  
\begin{align}
\label{randPhi} r(i,k)&\leftarrow\phi\, rand\\
\label{limitPhi}r_t^{(i,k)}&\leftarrow\begin{cases}
r(i,k)& \mathrm{if}\:r(i,k)<1\\
2-r(i,k)& \mathrm{otherwise}
\end{cases}\\
\label{targVel}Vdel_t^{(i,k)}[j]&=\begin{cases}
r_t^{(i,k)}& \mathrm{if}\: j \in del_t^{(i,k)}\\
0 &\mathrm{otherwise}
\end{cases}
\end{align}
the $\phi$ is a heuristic for tuning how aggressively the velocity points towards the  target particle k and here takes on a predetermined heuristic value between 0 and 2 (following tradition of continuous case). However, the velocity is limited to be in the range 0 to 1 since it is a probability of flipping, so values above 1 are reflected back. This doubles the density of $Vdel_t^{(i,k)}$ near 1 when  $\phi > 1$. The treatment of how to cope with $r(i,k) >1$ case is not well derived and  alternatives such as limiting all such cases to 1 may be better. The velocity increment encourages movement to the target $p_t^k$ by only contributing to the difference between $x_t^i$ and  $p_t^k$.
\subsubsection{Shotgun Blurring of Targets }
Typicaly there are a large number of elements (corresponding to dimensions in the continuous case) while the number of particles  is kept to just a few say at most 30. This means the use of just the velocity contributions directed to targets will explore only a small subset of changes. In the continuous case this has been solved in \cite{LcRiPSO} by adding a noise component to the velocity proportional to distance of particle from corresponding target, which effectively gives a shotgun effect. Further more this simple device makes the PSO converge to a local optimal solution.

For SPSO the distance is the Hamming distance between the particle and target given by
\begin{equation}\label{H-dist}
d_H(i,k)=|del_t^{(i,k)}|
\end{equation}
where $|z|$ is just the number of elements in $z$.this gives a blurring velocity
\begin{align}\label{blur-vel}
&b=rand(L_{factor}d_H(i,k)+L_{offset})/N\\
&Vblur_t^{(i,k)}[j]=b\:\; \forall j
\end{align}
where $L_{factor}$ and $L_{offset}$ are heuristics used to tune the shotgun effect.
\subsubsection{Inertia Term}
A heuristc $\omega$ borrowed from the continuous case is an inertia  term that multiplies the velocity probability by a reducing factor to slow down the progress of a particle and stop the velocity terms from over saturation.

Using the multiplying operation is crude and simple, which may not be the best; mapping from pseudo-addition to addition given in Equation \ref{add-map} suggests a more complex operation involving raising the probability of not flipping to some power. However, for the moment, simpler operation of multiplying which has the desired reduction of velocity is used here.
     
\subsubsection{Combining the Velocity Contributions}

these various contributions are combined to give
\begin{equation}\label{tot-velocity}
V_{t+1}^i=\omega\left (V_t^i\dot{+}\sum_{k \in N_t^i}^\bullet Vblur_t^{(i,k)}\right )\dot{+}\sum_{k \in N_t^i}^\bullet Vdel_t^{(i,k)}
\end{equation}
where multiplying by $\omega$ multiply each component of the velocity by $\omega$. In the continuous case  $\omega$ just multiplies $V_t^i$;
in the discrete case blurring tends to saturate and mask the update that moves the solution in direction of targets, so $\omega$ is applied to this as well.
         
\subsubsection{Updating to the Next Frame Using the Probabilistic Velocity}
In the continuous case the raw update is given by adding the velocity component to the current state. For the SPSO case by analogy we use flipping to give a raw update from the velocity,
thus put for each component:
\begin{equation}\label{raw_update}
(V_{t+1}^i[j]),\breve{x}_{t+1}^i[j]) \leftarrow\left\lbrace 
\begin{array}{cc}
(0,\neg x_t^i[j])&\mathrm{if}\:V_{t+1}^i[j]>rand \\
(V_{t+1}^i[j],x_t^i[j])&\mathrm{otherwise}
\end{array}\right.
\end{equation}
 

At this stage the raw state $\breve{x}_{t+1}^i$ may not satisfy the constraint so to keep things general a function $\mathtt{ToConstraint()}$\footnote{in the code this function has a diferent interface to minimise the API but is used to acheive the same thing.} supplied by the cost-function interface is applied  to attempt to make the raw state  satisfy the constraint based on the raw state as a starting hint and the old state,$x_t^i$ to facilitate this attempt.If it fails then revert back to the old state.Thus: 
\begin{align}
\breve{x}_{t+1}^i&\leftarrow&\mathrm{ToConstraint}(x_t^i,\breve{x}_{t+1}^i)\\
x_{t+1}^i &\leftarrow& \left\lbrace
\begin{array}{cc}
\breve{x}_{t+1}^i& \mathrm{if}\: \breve{x}_{t+1}^i \in \Omega\\
x_t^i&\mathrm{otherwise}
\end{array}\right.
\end{align} 
this gives an update that satisfies the constraints, $\Omega$ where  the initalisation of the state $x_0^i$ is chosen to ensure it also satisfies the constraint. this is done by random selection until the application of  \texttt{ToConstraint()} gives something that satisfies the constraint. Note during this operation the first argument will be the empty set and should be treated as a special case by the cost function.

\subsection{Updating the personal best}
To cope with noisy cost-functions the comparator function Comp() has the option of not being able to compare two  cases but will indicate which case needs to be further evaluated to improve the chance of a successful  comparison. let $x$  and $y$ be two constraint satisfying subset cases from this we get the cost function values $F(x)$ and $F(y)$ we now attempt a comparison say by using Comp() method on $F(x)$ and we use the  following symbolic notation depending on the result as follows:

\begin{tabular}{|c|c|c|}\hline
$F(x).\mathtt{Comp}(F(y))$ & Operator& Comment\\
\hline \hline 
0& $F(x)=F(y)$& \\
-1& $F(x)<F(y)$ &\\
1&$F(x)>F(y)$ &\\
-2&$F(x)\leftthreetimes F(y)$& $F(x)$ needs further evaluation\\
2&$F(x)\rightthreetimes F(y)$& $F(y)$ needs further evaluation\\
\hline	
	
\end{tabular}\newline
To get a comparison of costs between two subset cases $x$ and $y$  the optimiser  calls the cost function again when it needs  further evaluation and updates the cost  value with the new raw  value  from the cost function until there is enough drop in variance to do the comparison the \texttt{CostValue} is designed to ensure  that the costs are deemed equal if the variances cannot be reduced enough to discriminate between $F(x)$ and $F(y)$ so the comparison process is finite.



\subsection{grouping Swarm Particles } 
It is anticipated  that the main things that improve the SPSO performance is change in heuristics and the method used to choose the target sets $N_t^i$. Further more it is expected that particles in the swarm will adopt specialist behaviour by belonging to a group of particles. To support this the swarm is  partitioned into groups where each group of swarm particles has the same set of targets and heuristics. Groups are each given a name to aid describing a group.To begin with there is one group called "root" which contains all particles. Heuristics are often the same between groups so when forming a new group the heuristics are linked to the "root" group, although this can be modified and in general groups share heuristics to simplify changing a heuristic. Groups are populated by moving swarm particles from other groups thus maintaining a partition of swarm. By default the heuristics are chosen to have useful values and need not be explicitly given.              
  	
\section{How to deal with a noisy cost function?}

In the targeted application of Machine Learning the cost-function $F(x)$ is not a direct mathematical function of $x$ and has a random component so it has to be evaluated several times before comparing the fitness of $x$. 

The Particle Swarm Optimiser uses a personal best and global best
set of parameters and costs to guide it in its optimisation. in the
standard implementation this assumes that the cost function is deterministic
to the extent that the cost is assumed to be a direct function of
the parameters submitted to the cost function. However, this may not
be the case when dealing with real data where there are hidden parameters
that are not accounted for or the cost is a random function parametrised
by $x$. This means that the personal best may be due
to parameters with a fluke low cost which is rearly improved on during
the optimisation and thus cause the process to stick at a non optimal
value for typical behaviour. Furthermore the underlying statistics
may be slowly changing as the cost is sampled. 
\part{Miscellaneous Ideas}
\chapter{Statistics}
\section{Recursive formula for calculating expected values}

Let $S$ be a finite set and $P:2^S \longrightarrow [0,1]$ be the probability of subsets of $S$
Let $b_{i}$ be a sequence of independent samples of subsets of $S$ 
representing $P$. 

For a subset $A\subset S$ and a map $g:2^{A}\rightarrow\mathbb{R}$
we can get a sequence of estimates of expected values
\[
\mathbb{E}_{n}[g\parallel A]=\frac{1}{n}\sum_{i=1}^{n}g(b_{i}\cap A)
\]
of 
\[
\mathbb{E}[g\parallel A]=\sum_{B\subset S}P(B)g(B\cap A)
\]
which just calculates the expected value of $g$ acting on subsets
of $A$ . The double bar notation $\parallel$ is used to indicate
that we are looking on to the probability distribution through the
sub set window $A$. Now for $n>1$ 
\[
\mathbb{E}_{n}[g\parallel A]=\left(1-\frac{1}{n}\right)\mathbb{E}_{n-1}[g\mid A]+\frac{1}{n}g(b_{n}\cap A)
\]
we can generalise this to the sequence 
\[
\check{\mathbb{E}}_{n}[g\parallel A]=\left(1-\lambda_{n}\right)\mathbb{\check{E}}_{n-1}[g\parallel A]+\lambda_{n}g(b_{n}\cap A)
\]
with 
\[
\check{\mathbb{E}}_{1}[g\parallel A]=g(b_{1}\cap A)
\]
taking expected values and putting $\mathbb{\tilde{E}}_{n}[g\parallel A]=\mathbb{E}\left[\check{\mathbb{E}}_{n}[g\parallel A]\right]$
we get
\begin{align}
\mathbb{\tilde{E}}_{n}[g\parallel A] & = (1-\lambda_{n})\mathbb{\tilde{E}}_{n-1}[g\parallel A]+\lambda_{n}\mathbb{E}[g\parallel A]\\
& = (1-\lambda_{n})\mathbb{\mathbb{E}}[g\parallel A]+\lambda_{n}\mathbb{E}[g\parallel A]\\
& = \mathbb{E}[g\parallel A]
\end{align}

by induction on $n$. So even this sequence is an unbiased estimate
of the expected value. The important thing is to check for variance.
We have
\begin{align*}
\mathbb{E}\left[\left(\mathbb{\check{E}}_{n}[g\parallel A]-\mathbb{E}[g\parallel A]\right)^{2}\right] & = \mathbb{E}\left[\left(\mathbb{\check{E}}_{n}[g\parallel A]\right)^{2}\right]-2\mathbb{E}\left[\mathbb{\check{E}}_{n}[g\parallel A]\right]\mathbb{E}[g\parallel A]+\mathbb{E}[g\parallel A]^{2}\\
& = \mathbb{E}\left[\left(\mathbb{\check{E}}_{n}[g\parallel A]\right)^{2}\right]-\mathbb{E}[g\parallel A]^{2}\\
& = (1-\lambda_{n})^{2}\mathbb{E}\left[\left(\mathbb{\check{E}}_{n-1}[g\parallel A]\right)^{2}\right]+\lambda_{n}^{2}\mathbb{E}\left[g(b_{n}\cap A)^{2}\right]+(2\lambda_{n}(1-\lambda_{n})-1)\mathbb{E}[g\parallel A]^{2}\\
& = (1-\lambda_{n})^{2}\mathbb{E}\left[\left(\mathbb{\check{E}}_{n-1}[g\parallel A]\right)^{2}\right]+\lambda_{n}^{2}\mathbb{E}\left[g(b_{n}\cap A)^{2}\right]-((1-\lambda_{n})^{2}+\lambda_{n}^{2})\mathbb{E}[g\parallel A]^{2}\\
& = (1-\lambda_{n})^{2}\mathbb{E}\left[\left(\mathbb{\check{E}}_{n-1}[g\parallel A]-\mathbb{E}[g\parallel A]\right)^{2}\right]+\lambda_{n}^{2}\mathbb{E}\left[\left(g(b_{n}\cap A)-\mathbb{E}[g\parallel A]\right)^{2}\right]
\end{align*}
where we have used the independence of the $b_{i}$'s to give
\begin{align*}
\mathbb{E}\left[\mathbb{\check{E}}_{n-1}[g\parallel A]g(b_{n}\cap A)\right] & = \mathbb{E}\left[\check{\mathbb{E}}_{n-1}[g\parallel A]\right]E\left[g(b_{n}\cap A)\right]\\
& = \mathbb{E}[g\parallel A]^{2}
\end{align*}
put
\begin{align*}
\sigma^{2} & = \mathbb{E}\left[\left(g(b_{n}\cap A)-\mathbb{E}[g\parallel A]\right)^{2}\right]\\
\alpha_{n} & = \nicefrac{\mathbb{E}\left[\left(\mathbb{\check{E}}_{n}[g\parallel A]-\mathbb{E}[g\parallel A]\right)^{2}\right]}{\sigma^{2}}
\end{align*}
then we have
\[
\alpha_{n}=(1-\lambda_{n})^{2}\alpha_{n-1}+\lambda_{n}^{2}
\]
as a formulae that gives the change in variance of the estimates $\check{\mathbb{E}}_{n}[g\parallel A]$
.

As one can see there are several ways of combining measured means to give unbiased estimates of a noisy function's mean that gives a trade off on variance.
 


To directly calculate the expected cost at each iteration would be too expensive an operation especially if the parameters are way off optimal. Let $C(X)$ be the random variable representing the cost function for a given parameter vector $X$. for a given particle the current parameter changes rapidly while when it matters the personal best parameter, $X_b$ varies comparatively slowly so  when the $X_b$ is not updated one can take the opportunity to update the estimated expected cost function value $\hat{C}(X_b)$ of $\mathbb{E}[C(X_b)]$ in this case.This reduces the variance of the expected value compared to its raw value $C(X_b)$. 

Lets look at some ways in which the estimate in principle can be updated iteratively. Given samples $C_1,\cdots,C_i$ and a forgetting gain $\alpha$ we could produce an unbiased estimate
\begin{equation}
\label{est1}
\hat{C}(X_b)_i=\frac{\sum_{j=1}^i\alpha^{j-1}C_{i-j+1}}{\sum_{j=1}^i\alpha^{j-1}}
\end{equation}
Put 
\begin{align}
A_i&=\sum_{j=1}^i\alpha^{j-1}\\
\label{sum_a}
&=\frac{1-\alpha^i}{1-\alpha}
\end{align}
then using \eqref{est1} we get
\begin{align}
A_{i}\hat{C}(X_b)_{i}-\alpha A_{i-1}\hat{C}(X_b)_{i-1}=C_i
\end{align}
rearranging and using \eqref{sum_a} we get
\begin{align}
\hat{C}(X_b)_{i}&=\frac{\alpha A_{i-1}\hat{C}(X_b)_{i-1}+C_i}{A_{i}}\\
\label{est2}
&=\frac{\alpha(1-\alpha^{i-1})\hat{C}(X_b)_{i-1}+(1-\alpha)C_i}
{1-\alpha^{i}}
\end{align}
now if we put 
\begin{equation}\label{lam1}
\lambda_i=\frac{1-\alpha}{1-\alpha^{i}}
\end{equation}
we get substituting into \eqref{est2} the iterative update
\begin{equation}\label{est3}
\hat{C}(X_b)_{i}=(1-\lambda_i)\hat{C}(X_b)_{i-1}+\lambda_i C_i
\end{equation}
In fact any update of the form given in \eqref{est3} always gives an unbiased estimate  for arbitrary $\lambda_i$ . So we can use this as a generalised update. the choice of parameter is tuned to cater for how rapidly one wants to converge and the required variance of the result.

Looking at the variance update assuming independent samples. Put $\sigma_i^2$ as the variance of $\hat{C}(X_b)_{i}$ and $\sigma^2$ as the variance of $C_i$ then we get
\begin{equation}\label{vari1}
\sigma_i^2=(1-\lambda_i)^2\sigma_{i-1}^2 +\lambda_i^2 \sigma^2
\end{equation} 
The minimum value of the update variance as a function of $\lambda_i$ occurs at the point of inflexion given by
\begin{align}
0=&\frac{\partial \sigma_i^2}{\partial \lambda_i}\\
=&-2(1-\lambda_i)\sigma_{i-1}^2+2\lambda_i \sigma\\
=&2\{ \lambda_i(\sigma_{i-1}^2 + \sigma^2) -\sigma_{i-1}^2\}
\end{align} 
This gives the smallest variance $\lambda_i$ as
\begin{align}
\stackrel{*}{\lambda_i} &= \frac{\sigma^2}{\sigma_{i-1}^2 + \sigma^2}\\
&=\frac{1}{1+(\frac{\sigma^2}{\sigma_{i-1}^2})} \label{lam_st1}\\
&=\frac{\sigma^{-2}}{\sigma_{i-1}^{-2} + \sigma^{-2}}
\label{lam_st2}
\end{align}
From \eqref{lam_st2} we get
\begin{equation}\label{lam_st3}
1-\stackrel{*}{\lambda_i} =\frac{\sigma_{i-1}^{-2}}{\sigma_{i-1}^{-2} + \sigma^{-2}} 
\end{equation}
\eqref{lam_st1} show that the result depends only on the ratio of variances while \eqref{lam_st2} gives the result in form of inverse variance. using this gives easy to use results as will shortly be seen. Based in this remark substitute \eqref{lam_st2},\eqref{lam_st3} into \eqref{vari1} and calculate the updated inverse variance as
\begin{align}
\stackrel{*}{\sigma_i}^{-2}&=((1-\stackrel{*}{\lambda_i})^2\sigma_{i-1}^2 +\stackrel{*}{\lambda_i}^2 \sigma^{2})^{-1}\\
&=\frac{\sigma_{i-1}^{-2} \sigma^{-2}}{(1-\stackrel{*}{\lambda_i})^2\sigma^{-2} +\stackrel{*}{\lambda_i}^2 \sigma_{i-1}^{-2}}\\
&=\frac{\sigma_{i-1}^{-2}\sigma^{-2}(\sigma_{i-1}^{-2} + \sigma^{-2})^2 }{\sigma_{i-1}^{-4}\sigma^{-2} + \sigma_{i-1}^{-2}\sigma^{-4}}\\
&=\sigma_{i-1}^{-2}+\sigma^{-2} \label{inv_var1}
\end{align}  
This shows the inverse variance using the optimal $\lambda$ is additive and the expression is particularly simple so it can be calculated first and then $\stackrel{*}{\lambda_i}$ is evaluated from
\begin{equation}\label{lam_st4}
\stackrel{*}{\lambda_i}=\sigma^{-2}/\stackrel{*}{\sigma_{i}}^{-2}
\end{equation} 
This gives the minimum variance update. However in the long run as it repeats the $\lambda_i$ becomes so small that it ignores the current measurement and any change in the underlying statistics is ignored. We can avoid this by deliberately choosing a $\lambda_i$ which is larger than the optimal given by \eqref{lam_st4}. To this end put 
\begin{equation}\label{nlam1}
\lambda_i = (1+\delta)\stackrel{*}{\lambda_i}
\end{equation} 
then evaluate the updated inverse variance by substituting into
\eqref{vari1} the expression \eqref{nlam1} and simplifying using \eqref{inv_var1},\eqref{lam_st2} and \eqref{lam_st3}
\begin{align}
\sigma_i^{-2}&= \frac{\sigma^{-2} \sigma_{i-1}^{-2}}
{[(1-\stackrel{*}{\lambda_i})-\delta \stackrel{*}{\lambda_i}]^2\: \sigma^{-2}+\stackrel{*}{\lambda_i}^2 (1+\delta)^2\: \sigma_{i-1}^{-2}}\\
&= \frac{\sigma^{-2} \sigma_{i-1}^{-2}}{(1-\stackrel{*}{\lambda_i})^2\:\sigma^{-2}+ \stackrel{*}{\lambda_i}^2\:\sigma_{i-1}^{-2} 
	-2(1-\stackrel{*}{\lambda_i})\:\sigma^{-2}
	+2\stackrel{*}{\lambda_i}\sigma_{i-1}^{-2}
	+\stackrel{*}{\lambda_i}^2\delta^2(\sigma^{-2}+\sigma_{i-1}^{-2}) }	\\
&=\frac{\sigma^{-2}\sigma_{i-1}^{-2}\stackrel{*}{\sigma_i}^{-4} } 
{\sigma^{-2}\sigma_{i-1}^{-2}\stackrel{*}{\sigma_i}^{-2}+\sigma^-4 \stackrel{*}{\sigma_i}^{-2}\delta^2 }\\
&=\frac{\stackrel{*}{\sigma_i}^{-2}}
{1+\sigma^{-2}/\sigma_{i-1}^{-2}\:\delta^2 }\label{nlam2}
\end{align}

Equation \eqref{nlam2} is a particularly simple expression that shows how the inverse variance depends on $\delta$.


Most models of probability distribution start with combining pairs
of variables to give more elaborate models of the probability function.
Often in the case of neural nets the pairs are combined linearly
to give a single linear function which is then threshold before
being passed onto an output or hidden layer. Unfortunately this in
the case of neural nets gives rise to difficult to interpret models
with with often pathological behaviour. The aim here is to use the
multilinear approach that encourages more unique representation of
the data which can also be interpreted. However multi-linear functions
are only linear on each variable in isolation and are in general non
linear, hopefully in a useful way. When combining pairs of variables
we will combine to produce a result for a hidden variable (or output
from this ) and not threshold it. This section gives
results for combining pairs of variables. 

\section{Multi-linear functions and binary data probabilities}

Multi-linear expressions are a natural way of generating functions of
binary data. Let $S$ be a finite ordered set of size $n$ containing
the elements $x_{1},x_{2},\cdots,x_{n}$ then we can for each binary
sequence $a:S\rightarrow\{0,1\}$ of $n$ $0$ and $1$'s associate
a subset $\varphi(a)\subset S$ given as the elements $x_{i}$ of
$S$ such that the sequence $a$ maps $a(i)$ to $1$. Put for $A\subset S$
\[
Q(A)[x_{1},x_{2},\cdots,x_{n}]=\prod_{x_{i\in A}}x_{i}\prod_{x_{j}\notin A}(1-x_{j})
\]
Let $A,B\subset S$ such that $A\neq B$ then there exists an $k$
such that either $x_{k}\in A$ and $x_{k}\notin B$ or $x_{k}\in B$
and $x_{k}\notin A$ . assuming the former and let b be the associated
binary sequence of $B$. Evaluating $Q(A)$ with b we get that $b(k)=0$
so 
\[
Q(A)(b)\triangleq Q(A)(b(1),b(2),\ldots,b(n))=\prod_{x_{i\in A}}b(i)\prod_{x_{j}\notin A}(1-b(j))=0
\]
similarly for the other case. Also $Q(B)(b)=1$ by inspection. This
also shows that the $Q(A)[x_{1},x_{2},\cdots,x_{n}]$ are linearly
independent since if
\[
\sum_{A\subset S}c(A)Q(A)[x_{1},x_{2},\cdots,x_{n}]=0
\]
for coefficients $c(A)\in\mathbb{R}$ by substituting $a=\varphi^{-1}(A)$
we get
\[
c(A)=c(A)Q(A)(a)=0
\]
If we put for $A\subset S$
\[
\widetilde{Q}(A)[x_{1},x_{2},\cdots,x_{n}]=\prod_{x_{i\in A}}x_{i}
\]
then by induction on $n$ we can show that $Q(A)$ is a linear combination
of $\widetilde{Q}(A)$. Also by definition the $\widetilde{Q}(A)$
are independent and have the same number of dimensions $2^{n}$ so
span the same subspace of polynomials. Because of this the $Q(A)$
for $A\subset S$ span the space of \emph{multilinear polynomials}.

Further for each map $f:\,2^{S}\rightarrow\mathbb{R}$ we can associate
a a polynomial $Q(f)$ over the elements of $S$ given by
\[
Q(f)[x_{1},x_{2},\cdots,x_{n}]=\sum_{A\subset S}f(A)Q(A)
\]
and we get 
\[
Q(f)(b)=f(\varphi(b))
\]
This multi-linear or polynomial representation of the map $f$ is
in one to one correspondence so there is nothing added by using the
more complex polynomial representation of the map from subsets of
$S$ other than providing a smooth continuous representation of such
functions that can be easily approximated by simple combination of
polynomials that can be used to provide a more compact approximation
of $f$. In the general case $Q(f)$ is a large expression to evaluate
or store in a computer since it has $2^{n}$ terms. Also the coefficients
by themselves do not convey any useful structure or pattern without
further analysis. The question is can one introduce more restrictive
polynomial expressions that approximate $Q(f)$ and give easy interpretations
of the underlying patterns of sequences that generate significant
values of $f$ without significant loss of information.

In particular if $P(a)$ is the probability of the sequence $a$ then
$Q(P\circ\varphi^{-1})$ is a polynomial that maps binary sequences
to their respective probability. Call this the \emph{probability polynomial}
and Use the notation $Q_{P}[x_{1},x_{2},\cdots,x_{n}]$ for this.
The $\varphi$ map is an equivalence between subsets and binary sequences
so when not ambiguous leave it out. For instance $P(A)\triangleq P(\varphi^{-1}(A))$
for a subset $A$.

As a general rule low order polynomial representations avoid over
fitting so polynomials involving a small number of variables ( that
is elements of $S$ ) are useful. in particular combining variables
in pairs is useful especially if we create intermediate variables
called hidden  that represent linear polynomials of variables
and hidden variables. 
\subsection{bivariate binary table polynomial representation}
let $x$ and $y$ be binary valued variables. let $a:\{0,1\}\times\{0,1\}\rightarrow\{0,1\}$ be a function. this is  represented by a polynomial
\begin{align}
Q(a)[x,y]=&a(0,0)(1-x)(1-y)+a(0,1)(1-x)y + a(1,0)x(1-y)+a(1,1)xy \\
=&a(00)+[a(1,0)-a(0,0)]x+[a(0,1)-a(0,0)]y+[a(0,0)+a(1,1)-a(0.1)-a(1,0)]xy
\end{align} 
that agrees with the binary function $a$ for binary values. we can get a clean uncluttered parametrisation by putting
\begin{align}
X_0&=a(0,0)\\
X_1&=a(1,0)-a(0,0)\\
X_2&=a(0,1)-a(0,0)\\
x_3&=a(1,1)-a(0,0)\\
\end{align}   
to give 
\begin{equation}
\label{bin1}
Q(a)[x,y]=X_0+X_1x+X_2y+(X_3-X_1-X_2)xy
\end{equation}




\bibliographystyle{alpha}
\bibliography{AiPlayC.bib}



\end{document}